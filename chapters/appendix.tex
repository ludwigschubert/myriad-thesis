\chapter{Selected Source Code Examples}
\label{chapter:CodeExamples}

\begin{lstlisting}
  class Notification < ActiveRecord::Base
    belongs_to :user
    attr_accessible :user, :message, :resource_id, :resource_type

    def resource
      @resource ||= resource_class.find resource_id
    rescue ActiveRecord::RecordNotFound
      nil
    end
  
    def resource_class
      resource_type.classify.constantize
    end
  
    def has_resource?
      should_have_resource? and resource.present?
    end
  
    def should_have_resource?
      resource_type.present? and resource_id.present?
    end
   
  end

\end{lstlisting}

\chapter{UML Diagrams over time}
\label{chapter:UMLDiagrams}

Test esetnetcyiegslcsgeycagfjegsf

\section{Usage of UML Diagrams for internal communication}

\lipsum[2]

\section{Stability of UML Diagrams over time}

\lipsum[3]

\pagebreak

\insertfigure{figures/uml_class_diagram_01.pdf}{The first stable UML Diagram from January \nth{24} still had separate \code{Email} and \code{contacts\_messages} tables, no rule automation and a semantically incomplete model of Emails which didn't discern between user generated emails and system generated emails.}{UMLDiagram01}{1}

\insertfigure{figures/uml_class_diagram_02.pdf}{The second UML Diagram from May \nth{28} fixed most of the aforementioned problems and already introduced optimizations such as extracting the RawMail content to a different table. Those had become necessary when support for attachments was introduced.}{UMLDiagram02}{0.75}

\insertfigure{figures/uml_class_diagram_03.pdf}{This third UML Diagram from May \nth{29} finally tamed the \code{Email} inheritance tree, and also introduced improvements to legibility for the first time.}{UMLDiagram03}{1}

\insertfigure{figures/uml_class_diagram_04.pdf}{By the time of this fourth UML Diagram from April \nth{18} only minor improvements were still made to the system design, such as adding \code{ValueSettingAction}s and supporting assistants via \code{SharingAssignment}s.}{UMLDiagram04}{1}


\chapter{Colophon}
\label{chapter:Colophon}
This thesis is set in \LaTeX \cite{latex}. The template used is based on the official TUM Computer Science template. The sources are hosted publicly on GitHub, while the actual PDF file is built by the continuous integration server Travis.
