% Included by MAIN.TEX

\documentclass[11pt,a4paper,bibliography=totoc,index=totoc,headsepline,footsepline,footinclude=false,BCOR12mm,DIV13,parskip=half]{scrbook}

% KOMA-Optionen:
%  bibtotoc: include bibliography in table of contents
%  idxtotoc: include index in table of contents
%  headsepline: use horizontalline under heading
%  BCOR: binding correcion (Bindungskorrektur) (e.g.: BCOR5mm)
%  DIV: Number of sheet sections (used for layout) (e.g.: DIV12)

% Schriftart der Kopfzeile
\renewcommand{\sectfont}{\normalfont \bfseries}

% Use a sans-serif font. COmment out if you want a standard (serif) font.
\renewcommand{\familydefault}{\sfdefault}

% manipulate footer
\usepackage{scrpage2}
\pagestyle{scrheadings}
\ifoot[\footertext]{\footertext} % \footertext set in INFO.TEX
%\setkomafont{pagehead}{\normalfont\rmfamily}
% \setkomafont{pagenumber}{\normalfont\rmfamily}

\usepackage{setspace}
\linespread{1.5}

% Included by components/settings.tex

% Simply write the words you want hyphenated one per line with hyphens where you want them.


\usepackage{hyphenat}
\hyphenation{
par-a-digm
par-a-digms
mail-merge
Temp-late-Sent-To-Sear-ch-Re-sults-Ac-ti-vi-ty-No-ti-fi-ca-tion
Spread-sheet-Sync-ing-No-ti-fi-ca-ti-on
}

% Allows fancier image captions
\usepackage{caption}

% allow sophisticated control structures
% TeX is Turing complete, FYI.
\usepackage{ifthen}

% use Palatino as default font
%\usepackage{palatino}
\usepackage{helvet}

% enable special PostScript fonts
% \usepackage{pifont}

% make thumbnails
\usepackage{thumbpdf}

% to use the subfigures
\usepackage{subfigure}

% Use glossaries to make an index of abbreviations, technical words, etc.
\usepackage[acronym, section, nogroupskip, nopostdot]{glossaries}

% colorful tables I guess...
\usepackage{colortbl}

% Turns \nth{3} into 3rd with proper superscript.
\usepackage[super]{nth}

% Creates n paragraphs of Lorem Ipsum blindtext via \lipsum[n]
\usepackage{lipsum}

\usepackage[dvipsnames]{xcolor}

%% show program code\ldots
%\usepackage{verbatim}
%\usepackage{program}
\usepackage{listings}

\usepackage{fancyvrb}
\DefineShortVerb{\|}

% Allows the creation of footnotes that are listed at the end of a chapter via \endote{}
% Print the endotes bu including \theendnotes at the end of your chapter.
\usepackage{endnotes}

\lstset{ %
language=Ruby,                  % choose the language of the code
basicstyle=\footnotesize\ttfamily,       % the size of the fonts that are used for the code
commentstyle=\ttfamily\color{gray},
keywordstyle=\ttfamily\color{BrickRed}\bfseries\underbar,
stringstyle=\color{orange},
numbers=left,                   % where to put the line-numbers
numberstyle=\footnotesize,      % the size of the fonts that are used for the line-numbers
stepnumber=1,                   % the step between two line-numbers. If it is 1 each line will be numbered
numbersep=5pt,                  % how far the line-numbers are from the code
backgroundcolor=\color{white},  % choose the background color. You must add \usepackage{color}
showspaces=false,               % show spaces adding particular underscores
showstringspaces=false,         % underline spaces within strings
showtabs=false,                 % show tabs within strings adding particular underscores
frame=single,                   % adds a frame around the code
tabsize=2,                      % sets default tabsize to 2 spaces
captionpos=b,                   % sets the caption-position to bottom
breaklines=true,                % sets automatic line breaking
breakatwhitespace=false,        % sets if automatic breaks should only happen at whitespace
escapeinside={\%*}{*)}          % if you want to add a comment within your code
}

%% enable TUM symbols on title page
\usepackage{styles/tumlogo}


\usepackage{multirow}

%% use colors
\usepackage{color}

%% custom colors
\definecolor{SolarizedOrange}{HTML}{CB4B16}

% Format the section/chapter titles, e.g. with a different color
% \addtokomafont{sectioning}{\color{SolarizedOrange}}

%% make fancy math
\usepackage{amsmath}
\usepackage{amsfonts}
\usepackage{amssymb}
\usepackage{textcomp}
\usepackage{yhmath} % f¸r die adots
%% mark text as preliminary
%\usepackage[draft,german,scrtime]{prelim2e}

%% create an index
\usepackage{makeidx}

% for the program environment
\usepackage{float}

%% load german babel package for german abstract
%\usepackage[german,american]{babel}
\usepackage[german,english]{babel}
\selectlanguage{english}

% Use UTF8. Anything else is a crime nowadays.
\usepackage[utf8]{inputenc}

% use initals dropped caps - doesn't work with PDF
%\usepackage{dropping}
% Don't use the package as it doesn't seem to work by default.

\usepackage{styles/shortoverview}
%----------------------------------------------------
%      Graphics and Hyperlinks
%----------------------------------------------------

%% check for pdfTeX
\ifx\pdftexversion\undefined
 %% use PostScript graphics
 \usepackage[dvips]{graphicx}
 \DeclareGraphicsExtensions{.eps,.epsi}
 \graphicspath{{figures/}{figures/review}}
 %% allow rotations
 \usepackage{rotating}
 %% mark pages as draft copies
 %\usepackage[english,all,light]{draftcopy}
 %% use hypertex version of hyperref
 \usepackage[hypertex,hyperindex=false,colorlinks=false]{hyperref}
\else %% reduce output size \pdfcompresslevel=9
 %% declare pdfinfo
 %\pdfinfo {
 %  /Title (my title)
 %  /Creator (pdfLaTeX)
 %  /Author (my name)
 %  /Subject (my subject	)
 %  /Keywords (my keywords)
 %}
 %% use pdf or jpg graphics
 \usepackage[pdftex]{graphicx}
 \DeclareGraphicsExtensions{.jpg,.JPG,.png,.pdf,.eps}
 \graphicspath{{figures/}}

 %% Load float package, for enabling floating extensions
 \usepackage{float}

 %% allow rotations
 \usepackage{rotating}
 %% use pdftex version of hyperref
 \usepackage[pdftex,colorlinks=false,linkcolor=red,citecolor=red,%
 anchorcolor=red,urlcolor=red,bookmarks=true,%
 bookmarksopen=true,bookmarksopenlevel=0,plainpages=false%
 bookmarksnumbered=true,hyperindex=false,pdfstartview=%
 ]{hyperref}
%
%\usepackage[pdftex,colorlinks=false,linkcolor=red,citecolor=red,%
% anchorcolor=red,urlcolor=red,bookmarks=true,%
% bookmarksopen=true,bookmarksopenlevel=0,plainpages=false%
% bookmarksnumbered=true,hyperindex=false,pdfstartview=%
% ]{hyperref}
\fi

% Sets how many levels of Chapters/Sections/Subsections… etc. appear in the TOC
\setcounter{tocdepth}{2}

\usepackage[square,sort,comma,numbers]{natbib}

%% Fancy chapters
%\usepackage[Lenny]{fncychap}
%\usepackage[Glenn]{fncychap}
%\usepackage[Bjarne]{fncychap}

%\usepackage[avantgarde]{quotchap}

% set the bibliography style
\bibliographystyle{unsrt}
%\bibliographystyle{alpha}
%\bibliographystyle{plain}